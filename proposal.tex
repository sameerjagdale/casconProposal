\documentclass[a4paper, 12pt]{article}
\usepackage[bw]{mcode} % mcode listings
\newcommand{\matlab}{{\textsc Matlab}\space}
\newcommand{\velocty}{{Velo\textbf{C}ty}\space}
\newcommand{\smclab}{\textrm{\textsl{Mc}\textbf{\textsc{Lab}}}}
\newcommand{\mclab}{\smclab\space}
\begin{document}
\title{\velocty : An optimizing static compiler for array based languages}
\author{Sameer Jagdale}
\maketitle
The rising popularity of multi-core systems has renewed interest in the development of parallel algorithms. Research is also being carried out in the development of compiler tools to port existing systems to parallel architectures. Moreover, high-level scientific languages such as \matlab and Python with it's NumPy library are also gaining popularity among scientists and mathematicians. These languages provide many features such as a dynamic typing, functions like eval for runtime code evaluation, which allow easy prototyping. However the same features inhibit performance of the code. \\
We present \velocty, an optimizing static compiler for array based languages as a solution to the problem of enhancing performance of programs written in these languages. In most programs, a large portion of the time is spent executing a small part of the code. Hence, improved performance can be achieved by optimizing only specific 'hot' sections of the code. \velocty takes as input, functions written in \matlab and Python which are defined by the user as computationally intensive and generates an equivalent C++ version. It also generates glue code to interface with the source language. The generated code can then be compiled and packaged as a shared library that can be linked to any program written in the source language. \velocty also supports parallelism through OpenMP pragmas.\\
The compiler is part of the Velociraptor toolkit. It consists of a C++ backend for the velociraptor intermediate representation, VRIR, and language specific runtimes for \matlab and Python. We have also implemented language specific frontends for \matlab and Python which compile to VRIR. The \matlab frontend is implemented using the \mclab framework. \\
\velocty was evaluated using 16 \matlab benchmarks. The benchmarks versions using the C++ library were between 1.3 to 400 times faster than the pure \matlab versions. Experiments for the Python benchmarks were in progress at the time of writing of this abstract.  
\end{document}
