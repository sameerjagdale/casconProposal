\documentclass[a4paper, 12pt]{article}
\usepackage[bw]{mcode} % mcode listings
\newcommand{\matlab}{{\textsc Matlab}\space}
\begin{document}
\title{VeloCty : An optimising static compiler for array based languages}
\author{Sameer Jagdale}
\maketitle
The rising popularity of multi-core systems, has renewed interest in the development of parallel algorithms as well as compiler tools to port existing systems to parallel architectures. On the other side of the spectrum, high-level scientific languages such as \matlab and Python with it's NumPy library are also gaining popularity among scientists and mathematicians. These languages provide many features such as an interpreter style read print eval loop, functions like eval for runtime code evaluation and dynamic typing, which allow easy prototyping.However these languages also show poor performance due to the very same features. We present VeloCty, a optmising static compiler for array based languages as a solution to the problem of enhancing performance of programs written in these languages. \\ 
In most programs, a large portion of the time is spent executing a small part of the code. Hence improved performance can be  achieved by optmising only specific hot sections of the code. VeloCty takes as input functions written in \matlab and Python which are defined by the user as computationally intensive and generates an equivalent C++ version as well as glue code to interface with the source language. The generated code can then be compiled and packaged as a shared library that can we linked to any program written in the source language. This allows dynamic sections of the code which can not be trivially compiled to static languages to be represented in the source languages and also allows the users to continue programming in the language of their choice and yet observe significant speedups. 
\end{document}
