\documentclass[a4paper, 12pt]{article}
\usepackage[bw]{mcode} % mcode listings
\usepackage{xspace}
\usepackage{geometry}
\usepackage{a4wide}
\usepackage{parskip}
\usepackage{authblk}
\newcommand{\matlab}{{\textsc Matlab}\xspace}
\newcommand{\velocty}{{Velo\textbf{C}ty}\xspace}
\newcommand{\smclab}{\textrm{\textsl{Mc}\textbf{\textsc{Lab}}}}
\newcommand{\mclab}{\smclab\xspace}
\begin{document}
\title{\velocty : An optimizing static compiler for \matlab and Python}
\author[1]{Sameer Jagdale}
\affil[1]{ School of Computer Science, McGill University}
\maketitle
\section*{Abstract}
The rising popularity of multi-core systems has renewed interest in the development of parallel algorithms. Research is also being carried out in the development of compiler tools to port existing systems to parallel architectures. Moreover, high-level scientific languages such as \matlab and Python with its NumPy library are also gaining popularity among scientists and mathematicians. These languages provide many features such as a dynamic typing, functions like eval for runtime code evaluation etc. which allow easy prototyping. However these same features inhibit performance of the code. \\ 
We present \velocty, an optimizing static compiler for \matlab and Python as a solution to the problem of enhancing performance of programs written in these languages. In most programs, a large portion of the time is spent executing a small part of the code. Moreover, these sections can often be compiled ahead of time and improved performance can be achieved by optimizing only these `hot' sections of the code. \velocty takes as input functions written in \matlab and Python which are defined by the user as computationally intensive and generates an equivalent C++ version. \velocty also generates glue code to interface with the \matlab and Python. The generated code can then be compiled and packaged as a shared library that can be linked to any program written in the \matlab and Python. \velocty also supports parallelism through OpenMP.\\
\velocty uses the Velociraptor toolkit. It consists of a C++ backend for the Velociraptor intermediate representation, VRIR, and language-specific runtimes for \matlab and Python. We have also implemented language-specific frontends for \matlab and Python which compile to VRIR. The \matlab frontend is implemented using the \mclab framework. \\
\velocty was evaluated using 16 \matlab benchmarks. The benchmark versions using the C++ library were between 1.3 to 400 times faster than the MathWorks' \matlab2013a. Experiments for the Python benchmarks were in progress at the time of writing of this abstract.
\section*{Author's Biography}
Sameer Jagdale is a graduate student at McGill University. \velocty is his Master's thesis, being conducted at the Sable lab. He completed his undergraduate studies in Information Technology at the University of Pune, India.
\end{document}
